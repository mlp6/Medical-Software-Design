\documentclass[10pt]{report}
\usepackage{epsf}
\usepackage{amsmath}
\usepackage{amssymb}
\usepackage{palatino}
\usepackage[dvips]{graphics}
\usepackage{fancyhdr}
\usepackage{epsfig}
\usepackage{multirow}
\usepackage{multicol}
\usepackage{cancel}
\usepackage{hyperref}
\usepackage{longtable}
\parindent 0in
\parskip 1ex
\oddsidemargin  0in
\evensidemargin 0in
\textheight 8.5in
\textwidth 6.5in
\topmargin -0.25in

\pagestyle{fancy}
\lhead{\bf BME590.06: Medical Software Design}
\rhead{\bf Palmeri \& Kumar (Fall 2017)}
\cfoot{\thepage}


\begin{document}

\section*{Class Syllabus}

\subsection*{Instructors}

Dr. Mark Palmeri, M.D., Ph.D.\\
\href{mailto:mark.palmeri@duke.edu}{\nolinkurl{mark.palmeri@duke.edu}}\\
Office Hours: TBD (258 Hudson Hall Annex)

Suyash Kumar, CTO \& Co-Founder of Gradient Health\\
\href{mailto:suyash.kumar@duke.edu}{\nolinkurl{suyash.kumar@duke.edu}}\\
Office Hours: Tuesday 16:20-17:20 (Teer 106 or at
\href{http://chat.suyashkumar.com}{http://chat.suyashkumar.com})

Dr. David Ward, Ph.D.\\
\href{mailto:david.a.ward@duke.edu}{\nolinkurl{david.a.ward@duke.edu}}\\
Office Hours: Thursday 21:00-22:00 (Google Hangouts)

\emph{Google Hangout office hours can be joined by a URL that will be
sent to the class via email.}

\subsection*{Teaching Assistants}

Ismael Perez\\
\href{mailto:ismael.perez@duke.edu}{\nolinkurl{ismael.perez@duke.edu}}\\
Office Hours: Tuesday 14:00-15:00 (Gross Hall 3rd Floor Cubicle 36)

Tanvi Kamat Tarcar\\
\href{mailto:tanvi.kamat.tarcar@duke.edu}{\nolinkurl{tanvi.kamat.tarcar@duke.edu}}\\
Office Hours: Wednesday 11:20-12:20 (Teer Basement)

\subsection*{Lecture}

Tues/Thur 15:05-16:20\\
Teer 106

All lecture content will be outlined in \href{Lectures/}{Lectures}.

\subsection*{Course Overview}
Software plays a critical role in almost all medical devices, spanning device
control, feedback and algorithmic processing. This course focuses on software
design skills that are ubiquitous in the medical device industry, including
software version control, unit testing, fault tolerance, continuous
integration testing and documentation. Experience will be gained in Python
and JavaScript (and potentially other languages).

The course will be structured around a project to build an Internet-connected
medical device that measures and processes a biosignal, sends it to a web
server, and makes those data accessible to a web client / mobile application.
This project will be broken into several smaller projects to develop software
design fundamentals. All project-related work will be done in groups of 3
students.

\subsection*{Prerequisites}

Basic familiarity with programming concepts (e.g., variables, loops,
conditional statements).

\subsection*{Course Objectives}

\begin{itemize}
\item
  Software version control (\texttt{git}, GitHub)
\item
  Device programming fundamentals

  \begin{itemize}
  \item
    Review of data types, variables, loops, conditional statements
  \item
    Python (v3.6): numpy, scipy, pandas, scikit
  \item
    Virtual environments \& dependency management (\texttt{pip},
    \texttt{requirements.txt})
  \item
    Use of a programming IDE
  \item
    Debugging (\texttt{pudb})
  \end{itemize}
\item
  Testing

  \begin{itemize}
  \item
    Unit testing
  \item
    Functional / System testing
  \item
    Continuous integration (Travis CI)
  \end{itemize}
\item
  Fault tolerance (raising exceptions)
\item
  Logging
\item
  Resource profiling (\texttt{cProfile})
\item
  Documentation

  \begin{itemize}
  \item
    Docstrings
  \item
    Markdown
  \item
    Sphinx
  \item
    \href{https://readthedocs.org}{ReadTheDocs}
  \end{itemize}
\item
  Working with data

  \begin{itemize}
  \item
    Data Storage (Text, Binary, HDF5, MongoDB)
  \item
    Data Wrangling
  \end{itemize}
\item
  Data Processing \& Display

  \begin{itemize}
  \item
    Jupyter Notebooks
  \item
    Matplotlib / Seaborn
  \item
    Pandas (DataFrames)
  \item
    \href{https://scikit-image.org/}{scikit-image} \&
    \href{https://scikit-learn.org/}{scikit-learn}
  \end{itemize}
\item
  Define software specifications and constraints (Requests for Comments,
  RFC)
\item
  Servers

  \begin{itemize}
  \item
    Design \& Implementation of a biomedical web service (Python Flask)
  \item
    HTTP \& RESTful APIs
  \item
    Docker and dependency management intro
  \end{itemize}
\end{itemize}

\subsection*{Attendance}
Lecture attendance and participation is important because you will be working
in small groups most of the semester. Participation in these in-class
activities will count for 15\% of your class grade. It is very understandable
that students will have to miss class for job interviews, personal reasons,
illness, etc. Absences will be considered \textbackslash{}emph\{excused\} if
they are communicated to your instructors at least 48 hours in advance
(subject to instructor discretion as an excused absence) or, for illness,
through submission of a
\href{http://www.pratt.duke.edu/undergrad/policies/3531}{Short Term Illness
Form (STIF)} \textbf{before} class. Unexcused absences will count against the
participation component of your class grade.

\subsection*{Textbooks \& Resources}
There are no required textbooks for this class. A variety of online
resources will be referenced throughout the semester.

\begin{itemize}
\item
  \href{Resources/python.md}{Python Resources}
\end{itemize}

\subsection*{Project Details}
Project details will be discussed in lecture throughout the semester.

\subsection*{Grading}
The course GitHub repository will host all \href{Assignments/}{Assignments}.
Due dates--including those that change--will be announced in lecture and
by Sakai announcements that will be emailed to the class.

The following grading scheme is subject to change as the semester
progresses:

\begin{center}
\begin{tabular}{ll}
Participation	& 15\% \\
Assignments	& 35\% \\
Final project	& 50\% \\
\end{tabular}
\end{center}

\subsection*{Class Schedule}
The course schedule is very likely to change depending on progress
throughout the semester. The updated \href{schedule.md}{schedule} will
always be available in the GitHub course repository.

\begin{longtable}[c]{|l|p{0.4\textwidth}|p{0.4\textwidth}|}

  \hline 
  
  \textbf{Date} & \textbf{Lecture} & \textbf{Assignment} \\

  \hline

  Tues Aug 28 & Class Introduction, Objectives and Logistics & Setup Course Tools \& Git Fundamentals \\ \hline

  Thurs Aug 30 & Git: Repo Setup, Cloning/Forking, Issues, Branching, Pushing/Pulling & \\ \hline
  
  Tues Sept 04 & Git Workflow & Getting Started with git \\ \hline
  
  Thurs Sept 06  & Python Virtual Environments & \\ \hline
  
  Tues Sept 11 & Python Fundamentals & \\ \hline
  
  Thurs Sept 13 & Class Cancelled (Severe Weather) & \\ \hline
  
  Tues Sept 18 & NO LECTURE (NC State Career Fair) & \\ \hline

  Thurs Sept 20 & Unit Testing: (py.test) \& Continuous Integration (Travis CI) & \\ \hline
  
  Tues Sept 25 & Unit Testing: Comprehensive unit tests & Unit Testing \& Continuous Integration (Travis CI) \\ \hline
  
  Thurs Sept 27 & IEC 62304 &  \\ \hline
  Tues Oct 02 & Unit Testing: Approximations, fixtures \& more; Docstrings & 
  PythonFundamentals.ipynb (Sakai) \& IEC 62304 Assessment (Sakai) \\ \hline
  
  Thurs Oct 04 & \href{Resources/pycharm.md}{PyCharm}; Debugging; Property Decorators & \\ \hline
  
  Tues Oct 09 & Fall Break & \\ \hline
  
  Thurs Oct 11 & Functional Decomposition; Python: Data Structures & Heart Rate Monitor \\ \hline
  
  Tues Oct 16 & Exceptions \& Logging & \\ \hline
  
  Thurs Oct 18 & Dictionary Type, Classes, Property Decorators, Numpy Docs & \\ \hline
  
  Tues Oct 23 & HRM Assignment Work & \\ \hline
  
  Thurs Oct 25 & HRM Assignment Work & \\ \hline
  
  Tues Oct 30 & Intro to Web Services \& Cloud-connected Devices & \\ \hline
  
  Thurs Nov 01 & Python Flask, API design, virtual machines (Duke OIT VMs) & 
  Call web services (SendGrid, Twilio, etc) \\ \hline
  
  Tues Nov 06 & Flask continued, deployment, production considerations & \\ \hline
  Thurs Nov 08 & Introduction to Databases & Heart Rate Sentinel Server \\ \hline
  Tues Nov 13 & Introduction to Security + Assignment Work & TBD \\ \hline
  Thurs Nov 15 & Working Project Code & \\ \hline
  Tues Nov 20 & Final project assignment & \\ \hline
  Thurs Nov 22 & Thanksgiving & Refactor Project Code \\ \hline
  Tues Nov 27 & Final project work \\ \hline
  Thurs Nov 29 & LDOC! \\ \hline
  Tues Dec 04 & Final project work & \\ \hline
  Thurs Dec 13 & Final project due & \\ 

  \hline

\end{longtable}


\subsection*{Distributed Version Control Software (git)}
Software management is a ubiquitous tool in any engineering project, and this
task becomes increasingly difficult during group development. Version control
software has many benefits and uses in software development, including
preservation of versions during the development process, the ability for
multiple contributors and reviewers on a project, the ability to tag
\emph{Releases} of code, and the ability to branch code into different
functional branches. We will be using \href{https://github.com}{GitHub} to
centrally host our git repositories. Specifically, we will be creating
student teams in the \href{https://github.com/Duke-BME-Design}{Duke BME
Design} group. Some guidelines for using your git repositories:

\begin{itemize}
\item
  \emph{All} software additions, modifications, bug-fixes, etc. need to
  be done in your repository.
\item
  The \emph{Issues} feature of your repository should be used as a "to
  do" list of software-related items, including feature enhancements,
  and bugs that are discovered.
\item
  There are several repository management models that we will review in
  class, including branch-development models that need to be used
  throughout the semester.
\item
  Instructors and teaching assistants will only review code that is
  committed to your repository (no emailed code!).
\item
  All of the commits associated with your repository are logged with
  your name and a timestamp, and these cannot be modified. Use
  descriptive commit messages so that your group members, instructors,
  and teaching assistants can figure out what you have done!! You should
  not need to email group members when you have performed a commit; your
  commit message(s) should speak for themselves.
\item
  Code milestones should be properly tagged.
\item
  Write software testing routines early in the development process so
  that anyone in your group or an outsider reviewing your code can be
  convinced that it is working as intended.
\item
  Modular, modular, modular.
\item
  Document!
\item
  Make commits small and logical; do them often!
\end{itemize}

We will review working with git repositories in lecture, and feedback on
your software repository will be provided on a regular basis.

\subsection*{Online Slack Channels}
We have online help through the \href{https://dukecolab.slack.com/}{Duke
Co-Lab Slack} team. We have started three specific channels for this
class: \texttt{\#linux}, \texttt{\#git} \& \texttt{\#python}. Please add
yourselves to these channels to get help from your instructors, your TAs
and the Duke community!

\subsection*{Duke Community Standard \& Academic Honor}
Engineering is inherently a collaborative field, and in this class, you
are encouraged to work collaboratively on your projects. The work that
you submit must be the product of your and your group's effort and
understanding. All resources developed by another person or company, and
used in your project, must be properly recognized.

All students are expected to adhere to all principles of the
\href{http://www.integrity.duke.edu/standard.html}{Duke Community
Standard}. Violations of the Duke Community Standard will be referred
immediately to the Office of Student Conduct. Please do not hesitate to
talk with your instructors about any situations involving academic
honor, especially if it is ambiguous what should be done.

\end{document}