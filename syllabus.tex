\documentclass[10pt]{report}
\usepackage{epsf}
\usepackage{amsmath}
\usepackage{amssymb}
\usepackage{palatino}
\usepackage[dvips]{graphics}
\usepackage{fancyhdr}
\usepackage{epsfig}
\usepackage{multirow}
\usepackage{multicol}
\usepackage{cancel}
\usepackage{hyperref}
\usepackage{longtable}
\parindent 0in
\parskip 1ex
\oddsidemargin  0in
\evensidemargin 0in
\textheight 8.5in
\textwidth 6.5in
\topmargin -0.25in

\pagestyle{fancy}
\lhead{\bf BME590.06: Medical Software Design}
\rhead{\bf Palmeri \& Kumar (Fall 2017)}
\cfoot{\thepage}


\begin{document}

\section*{Class Syllabus}

{\bf Instructor:} Dr. Mark Palmeri \href{mailto:mark.palmeri@duke.edu}{(mark.palmeri@duke.edu)}

{\bf Teaching Assistant:} Brenton Keller\href{mailto:brenton.keller@duke.edu}{(nbb5@duke.edu)}

{\bf Lecture:} Tues/Thurs 08:30--09:40, Fitzpatrick Schiciano B 1466

\subsection*{Course Overview}
Software plays a critical role in almost all medical devices, spanning device control, feedback and algorithmic processing.  This course focuses on software design skills that are ubiquitous in the medical device industry, including software version control, unit testing, fault tolerance, integration testing and documentation.  Experience will be gained in both dynamically- (Python) and statically-typed (C/C++) languages.  Efficient programming practices for low computational resource devices will be covered.

The course will be structured around several small projects working with biosignals to develop software design fundamentals.  The course will culminate in a larger project working with clinical/research data and/or designing software for a small, embedded device.  Students will be expected to work in small groups.

Prerequisites: Introductory Programming Class (e.g., EGR103)

\section*{Course Objectives}
\begin{multicols}{2}
\begin{itemize}
    \item Define software specifications and constraints
    \item Device programming fundamentals
    \begin{itemize}
        \item Review of data types
        \item Analog-to-digital / digital-to-analog conversion
        \item Python (v3.5): numpy, scipy, pandas, scikit
        \item C/C++
        \item Simplified Wrapper and Interface Generator (SWIG)
        \item Data management (variables, references, pointers, ASCII/Unicode/binary data)
        \item Compilation, make, cmake
    \end{itemize}
    \item Software version control (git, GitLab)
    \item Project management (Redmine)
    \item Biosignals
    \begin{itemize}
        \item Anatomy/physiology review
        \item Signal transduction
        \item Noise \& artifacts
    \end{itemize}
    \item Signal processing
    \begin{itemize}
        \item Convolution \& correlation
        \item Filtering
        \item Peak detection
        \item Envelope detection
        \item Spectral analysis
        \item Wavelet analysis
    \end{itemize}
    \item Documentation
    \begin{itemize}
        \item Docstrings
        \item Markdown
        \item Sphinx / Doxygen
    \end{itemize}
    \item Testing
    \begin{itemize}
        \item Unit testing
        \item System testing
        \item Continuous integration (GitLab Runners)
    \end{itemize}
    \item Fault tolerance (raising exceptions)
    \item Resource profiling (cProfile)
    \item Debugging
\end{itemize}
\end{multicols}

\subsection*{Fall Semester Class Schedule} 
All class activities will be officially listed in the class Sakai site
calendar.  Specific lecture details, along with deliverable due dates,
will be posted on this calendar, and new due dates will be announced in lecture
and by Sakai announcements that will be emailed to the class.  The following is
a summary of activities this semester (subject to change):

\begin{longtable}[c]{|l|l|l|}

    \hline 
    
    \textbf{Date} & \textbf{Lecture} & \textbf{Assignment}\\

    \hline

    Tues Aug 29     & Class Introduction, Objectives and Logistics; Git Demo & A01: Git(Hub), Python\\
    Thurs Aug 31    & Git: Repo Setup, Issues, Branching, Pushing/Pulling & In-class exercise\\
    \hline
    Tues Sep 05     & Python virtualenv (conda/pip), Unit Testing & In-class exercise\\
    Thurs Sep 07    & Continuous Integration (Travis CI), Python Types/Exceptions & Comlete unit test exercise\\
    \hline
    Tues Sep 12     & Python: Dictionaries, Numpy Arrays, Binary Data & A02: Heart Rate Monitor \\
    Thurs Sep 14    & Python: Project Structure (approaching HRM) & \\
    \hline
    Tues Sep 19     & Data Types, Application Program Interfaces (API) \& JSON & \\
    Thurs Sep 21    & Modules, Classes, Composition& \\
                    & Classes: Inheritance \& Composition & \\
    \hline
    Tues Sep 26     & Python: docstrings, Sphinx & A03: TBD\\
    Thurs Sep 28    & Python: try/except, logging & \\
    \hline
    Tues Oct 03     & Python: Read/Writing Data (CSV, JSON, HDF5, MATv5) & A04: robust, logging \\
    Thurs Oct 05    & Regular Expressions & \\
    \hline
    Tues Oct 10     & FALL BREAK & \\
    Thurs Oct 12    & Installing SWIG \& C/C++ Compiler & \\
    \hline
    Tues Oct 17     & Python $\rightarrow$ C/C++ \& Static-Compilation & \\
    Thurs Oct 19    & Binary Data Files \& Bit Operations & \\
    \hline
    Tues Oct 24     & Continuous Integration Testing & \\
    Thurs Oct 26    & Regular Expressions & \\
    \hline
    Tues Oct 31     & TBD & A05: TBD \\
    Thurs Nov 02    & Simplified Wrapper \& Interface Generator (SWIG) & \\
    \hline
    Tues Nov 07     & TBD & A06: TBD \\
    Thurs Nov 09    & TBD & \\
    \hline
    Tues Nov 14     & IEC 62304 & Project Unit Tests \\
    Thurs Nov 16    & Project TDD & \\
    \hline
    Tues Nov 28     & Debugging & Working Project Code \\
    Thurs Nov 30    & Presentations: Profiled Code & \\
    \hline
    Tues Dec 05     & Project ``Lab'' & Refactor Project Code \\
    Thurs Dec 07    & Final Project Due & \\
    \hline

\end{longtable}


\subsection*{Attendance}
You will be part of a design team, and your participation in group activities
is essential.  Lecture / lab attendance and participation count for 10\% of
your class grade.  It is very understandable that students will have to miss
class / lab for job interviews, personal reasons, illness, etc.  Absences will
be considered \emph{excused} if they are communicated to Dr. Palmeri at least
48 hours in advance or through submission of
\href{http://www.pratt.duke.edu/undergrad/policies/3531}{Short Term Illness
    Form (STIF)} {\bf before} class or lab.  Unexcused absences will count
against the attendance and participation component of your class grade.  Peer
evaluations throughout the semester will also count towards your participation
grade.

\subsection*{Office Hours}
Dr. Palmeri will have office hours for an hour after lectures on Tuesdays and
Thursdays from 14:40--15:40.  Other days / times are always available by
appointment, and there is an open-door policy active all semester (i.e., if my
office door is open, then you are welcome to pop in).  Nick and Matt will also
be available, as needed, to help with all aspects of your projects throughout
the semester.

\subsection*{Textbooks \& References} 
There are no required textbooks for this class.  However, given the diverse background of everyone, textbooks may play an essential role in complementing lecture material.  Please consider the following:

\begin{description}
    \item[Electronics] The Art of Electronics by Horowitz and Hill, Cambridge,
        Third Edition (ISBN 978-0-521-80926-9) Arduino, then I recommend that
        you consider looking at Jeremy Blum's
    \item[Analog \& Digital Circuits] Practical Electronics for Inventors by
        Scherz and Monk, Third Edition (ISBN 978-0-07-177133-7)
    \item[Medical Devices] Medical Device Technologies by Baura, Academic Press
        (ISBN 978-0-12-374976-5)
    \item[Biomedical Device Design] Design of Biomedical Devices and Systems by
        King, Fries and Johnson, CRC Press, Third Edition (ISBN
        978-1-4665-6913-3)
\end{description}

\subsection*{Electronic Lab Notebooks (ELN)} 
GitLab will be used to host a single repository for each project that will act
as the ELN.  Your ELN is one of your primary ways to demonstrate and document
your entire design process, including brainstorming of design ideas, formal
engineering analysis to justify your design choices, testing procedures, and
performance analysis.  It can also serve as a legal document for intellectual
property, etc.  Your ELN / documentation will account for 20\% of your grade.

Some guidelines for your ELN:

\begin{itemize}
    \item Block diagrams of the device and submodules in the device should be
        presented.
    \item Signal processing and software algorithm logic should be documented,
        and the associated code should be specifically referenced by the
        appropriate git commit SHA hash(es).
    \item Choice of electronic components should be supported by an analysis
        (when appropriate) that includes considerations such as power,
        bandwidth, SNR, etc.\ that could impact performance of your device.
    \item Formal circuit schematics should be drawn and properly annotated for
        all electronic circuits.
    \item Component datasheets, user manuals, etc.\ should be included.
    \item Mechanical drawings, SolidWorks renderings, etc.\ should be included.
    \item All measurements made during debugging and testing should be recorded
        with the appropriate precision, units, and estimates of uncertainty.
        These measurements should be accompanied by a motivation for making
        them, along with a summary of your interpretation of the data.
    \item A copy of all reports and slides should be saved.
    \item All software associated with your device, along with relevant
        documentation, should be maintained in your group's git repository.  
    \item Git tags should be used to delineate project milestones.
\end{itemize}

We will review ELN content in lecture, and your group will receive regular
feedback.

\subsection*{Distributed Version Control Software (git)} 
Software management is a ubiquitous issue in any microcontroller engineering
project, and this task becomes increasingly difficult during group development.
Version control software has many benefits and uses in software development,
including preservation of versions during the development process, the ability
for multiple contributors and reviewers on a project, the ability to tag
``releases'' of code, and the ability to branch code into different functional
branches.  We will be using GitLab (\url{https://gitlab.oit.duke.edu}) to
centrally host our git repositories, with one repository existing for each
project.  Some guidelines for using your git repositories:

\begin{itemize}
    \item \emph{All} software additions, modifications, bug-fixes, etc.\ need
        to be done in your repository.
    \item The ``Issues'' feature of your repository should be used as a ``to
        do'' list of software-related items, including feature enhancements,
        and bugs that are discovered.
    \item There are several management models that we will review in class,
        with the major distinctions being individual user forks of the master
        repository, or a single mutli-write-user repository.  We will review
        these different management models in class, after which your group
        should choose one and stick with it.
    \item Dr. Palmeri will only review code that is committed to your master
        repository or your personal fork of that repository.  
    \item All of the commits associated with your repository are logged with
        your name and a timestamp, and these cannot be modified.  Use
        descriptive commit messages so that your group members, Dr. Palmeri,
        Matt Brown and Nick can figure out what you have done!!  You should not
        need to email group members when you have performed a commit; your
        commit message(s) should speak for themselves.
    \item Code milestones should be properly tagged.
    \item Write software testing routines early in the development process so
        that anyone in your group or an outsider reviewing your code can be
        convinced that it is working as intended.
    \item Comment, comment, comment.
    \item Modular, modular, modular.
    \item Make commits small and logical; do them often!
\end{itemize}

We will review working with git repositories in lecture and lab, and feedback
on your software repository will be provided on a regular basis.

\subsection*{Project Details} 
Project details, including budgets, part ordering procedures, etc.\ will be
detailed in lecture with a separate handout.

\subsection*{Grading} 
The following grading scheme is subject to change as the semester progresses:

\begin{center}
\begin{tabular}{ll}
Attendance \& Participation             & 10\% \\
ELN / Documentation                     & 20\% \\
Technical Reports / IRB / IDF           & 20\% \\
Presentations                           & 20\% \\
Final Device (Design \& Function)       & 30\% \\
\end{tabular}
\end{center}

\emph{\bf Since this is a 2 semester design class, grades will be recorded only
    after the completion of both semesters!!}

\subsection*{Duke Community Standard \& Academic Honor} Engineering is
inherently a collaborative field, and in this class, you are encouraged to work
collaboratively on your projects.  The work that you submit must be the product
of your and your group's effort and understanding.  All resources developed by
another person or company, and used in your project, must be properly
recognized.
 
All students are expected to adhere to all principles of the
\href{http://www.integrity.duke.edu/standard.html}{Duke Community Standard}.
Violations of the Duke Community Standard will be referred immediately to the
Office of Student Conduct.  Please do not hesitate to talk with Dr.\ Palmeri
about any situations involving academic honor, especially if it is ambiguous
what should be done.

\end{document}
